\documentclass{article}
\usepackage{arxiv}

\usepackage[utf8]{inputenc}
\usepackage[english, russian]{babel}
\usepackage[T1]{fontenc}
\usepackage{url}
\usepackage{booktabs}
\usepackage{amsfonts}
\usepackage{nicefrac}
\usepackage{microtype}
\usepackage{lipsum}
\usepackage{graphicx}
\usepackage{natbib}
\usepackage{doi}



\title{A template for the \emph{arxiv} style}

\author{ Владислав Пыж	\\
	МФТИ\\
	Долгопрудный, Россия \\
	\texttt{pyzh.va@phystech.edu} \\
	%% examples of more authors
	\And
	Юрий Максимов\\
	\\
	\\
	\texttt{} \\
	%% \AND
	%% Coauthor \\
	%% Affiliation \\
	%% Address \\
	%% \texttt{email} \\
	%% \And
	%% Coauthor \\
	%% Affiliation \\
	%% Address \\
	%% \texttt{email} \\
	%% \And
	%% Coauthor \\
	%% Affiliation \\
	%% Address \\
	%% \texttt{email} \\
}
\date{}

\renewcommand{\shorttitle}{\textit{arXiv} Template}

%%% Add PDF metadata to help others organize their library
%%% Once the PDF is generated, you can check the metadata with
%%% $ pdfinfo template.pdf
\hypersetup{
pdftitle={A template for the arxiv style},
pdfsubject={q-bio.NC, q-bio.QM},
pdfauthor={David S.~Hippocampus, Elias D.~Striatum},
pdfkeywords={First keyword, Second keyword, More},
}

\begin{document}
\maketitle

% \begin{abstract}
В данной работе рассматривается задача предсказания экстремальных климатических явлений, а именно наводнений. Прогнозирование осуществляется в краткосрочном диапозоне для стационарных временных рядов и в длинном диапозоне для нестационарных рядов. Существенной особенностью данной задачи является необходимость предсказывания экстремальных значений с высокой точностью, тогда как точность предсказаний малых изменений не представляет интерес.
% \end{abstract}


\end{document}